\documentclass{beamer}
\title{Heritability of Protein Phenotypes}
\author{Alvaro Barbeira}
\usepackage[utf8]{inputenc}
\usepackage{default}
\usetheme{Warsaw}

\begin{document}
  %title
  \begin{frame}
    \titlepage
  \end{frame}
  \begin{frame}
    \frametitle{Outline}
    \tableofcontents
  \end{frame}
  %intro
  \section{Problem Introduction}
  \begin{frame}
    \frametitle{Problem:}
    \framesubtitle{Explore heritability of Proteins}
    \begin{itemize}
      \item
      Heritability of proteins was chosen as study subject because there are not many published results about it.
      \item
      Low data quality and noise was expected.
      \item
      Chosen data sets: hapmap release 23, Protein Data from Ron Hause and Lingfeng Wu
    \end{itemize}
  \end{frame}
  
  \section{Heritability}
  \begin{frame}
    \frametitle{General Method}
    The basic procedure was:
    \begin{itemize}
      \item
      Select individuals that satisfied $MAF >= 0.05$ with plink 1.07
      \item
      Generate Genetic Relationship Matrix with GCTA 1.24.4
      \item
      Use GCTA to figure out Restricted Maximum Likelihood (REML) for each protein
    \end{itemize}
    
    Results:
  \end{frame}
\end{document}
